\chapter{Pendahuluan}
\section{Section example}
This is an example of a section of a text. It uses the regular indentation for the first line.

\section{Tables}
Tables use the \texttt{longtblr} environment so it can handle multi-page tables. 
\begin{longtblr} 
[caption={Table example},
label={tab:exampleTable}]
	{
		colspec={ c c c }, % Column type
		rowhead = 1, % Header row
		row{1} = {font=\bfseries\sffamily}, % Header row styling
		hlines,	vlines % Table lines
	}
	First & Second & Third \\
	1 & 2 & 3
\end{longtblr}


\section{Images}
Images will make use of the \texttt{images} environment with the [H] float in order to anchor it to its position in the page.
\begin{figure}[H]
	\centering
	\includegraphics{\detokenize{SANATA_DHARMA.jpg}}
	\caption{Image Example}
	\label{fig:exampleFigure}
\end{figure}

\subsection{Subimages}
Multiple images can be had in a single figure environment using the subfigure environment.
\begin{figure}[H]
	\centering
	\begin{subfigure}{0.4\textwidth}
		\centering
		\includegraphics{\detokenize{SANATA_DHARMA.jpg}}
		\caption{First Figure}
	\end{subfigure}
	\begin{subfigure}{0.4\textwidth}
		\centering
		\includegraphics{\detokenize{SANATA_DHARMA.jpg}}
		\caption{Second Figure}
	\end{subfigure}
	\caption{Subfigures.}
\end{figure}

\section{Labeling}
Some environments can be labeled, such as tables and figures. There are several things to consider when creating labels. \texttt{longtblr} environments use parameters to define labels, which can be seen in the example below.

\begin{lstlisting}
\begin{longtblr} 
	[caption={Table example},
	 label={tab:exampleTable} <---------LABEL
	]
	{
		colspec={ c c c }, % Column type
		rowhead = 1, % Header row
		row{1} = {font=\bfseries\sffamily}, % Header row styling
		hlines,	vlines % Table lines
	} 
	First & Second & Third \\
	1 & 2 & 3
\end{longtblr}
\end{lstlisting}

\noindent which can be refered by calling \texttt{\detokenize{\ref{tab:exampleTable}}}. When called, it will look something like this: \ref{tab:exampleTable}. The reference does not contain the name of the object being refered, only the number. Therefore, it is best to include the name of the object, such as: Table \ref{tab:exampleTable}.

Images are labeled using the \texttt{\detokenize{\label{}}} command. An example is shown below:

\begin{lstlisting}
\begin{figure}[H]
	\centering
	\includegraphics{\detokenize{SANATA_DHARMA.jpg}}
	\caption{Image Example}
	\label{fig:exampleFigure} <----------------LABEL
\end{figure}
\end{lstlisting}

\noindent where refering to said label can be done with the afformentioned \texttt{\detokenize{\ref{}}} command. Subfigures can also be refered in this manner, so make use of it when you wish to do so.